\documentclass{beamer}
\usepackage[T1]{fontenc}
\usepackage[utf8]{inputenc}
\usepackage{pslatex}
\usepackage[greek.polutoniko,hebrew,english]{babel}
\usepackage{cjhebrew}
\usetheme{Berkeley}
\usecolortheme{beetle}

\title{Free Software and Christian Theology I}
\author{Kenneth Gardner}

\begin{document}

\maketitle

\section{Introduction}

\begin{frame}
  ``Free Software'' is a technical designation.
  It does not denote price, but it frequently \emph{is} free in that sense.
  It denotes ``freedom'' as in ``liberty.''
  I have bought free software.
\end{frame}

\begin{frame}
  To avoid the confusion, the term has been changed to ``libre software.''\\ \pause
  Another designation, ``open source'' exists, but that's a hornets' nest.\\ \pause
  It has been abbreviated to FOSS (free/open source software).
\end{frame}

\begin{frame}
  Traditionally these freedoms are broken down into four categories:\pause
  \begin{itemize}
	\item Freedom 0: Run it as you want.\pause
	\item Freedom 1: See the source and modify it.\pause
	\item Freedom 2: Distribute copies of the program to help others.\pause
	\item Freedom 3: Distribute copies of your modifications or modified versions.\pause
  \end{itemize}
  In other words, the same kinds of rights you get when you buy, say, a chair.
\end{frame}

\begin{frame}
  Free software licenses may designate that you must make all derivative products under the same license.
  Other license turn away all control.

  It is applying a real concept of ownership to computers.
  You can own the software you buy.
\end{frame}

\begin{frame}
  Compare this with other EULAs.\pause
  \begin{itemize}
	\item You may not copy the software.\pause
	\item You may not modify the software.\pause
	\item You may not reverse engineer the software.\pause
	\item This license may be revoked at any time by the publisher.\pause
	\item And many more
  \end{itemize}
\end{frame}

\begin{frame}
  Meaning of restrictions:\pause
  \begin{itemize}
	\item You cannot own it.\pause
	\item You are subject to arbitrary power.\pause
	\item If anything evil is lurking you usually cannot know.\pause
	\item If you do, you cannot do anything about it.\pause
	\item There are more that we can line out.
  \end{itemize}
\end{frame}

\section{Jesus and Money}

\begin{frame}
  Jesus' teachings on money are both demanding and almost impossible to fulfill.
\end{frame}

\begin{frame}
  The Sermon on the Mount laid out some fairly stark demands:\pause
  \begin{itemize}
	\item If someone compels you to walk a mile, walk two (Mt 5.41)\pause
	\item If someone takes your outer coat, give your inner also (Mt. 5.40)\pause
	\item Give to the one who asks, and do not ask back (Mt 5.42)\pause
	\item Don't store wealth on earth but in heaven (Mt 6.19)\pause
	\item ``You cannot serve God and money'' (Mt 6.24c)
  \end{itemize}
\end{frame}

\begin{frame}
  These demands are almost unfulfillable.
  We will fail, and I have on several occasions.
  Those, however, are the criteria he laid down.
\end{frame}

\begin{frame}
  Luke's Sermon on the Plain was even more savage:
  \begin{itemize}
	\item Woe to you who are rich, because you have received your comfort.
	\item Woe to you who are full now, because you will hunger.
  \end{itemize}
\end{frame}

\begin{frame}
  Luke also wrote Acts as part of the same text.
  In it, he makes a point that the early disciples shared everything:\\
  All those who were faithful were together and held all things in common,
  and they were selling their properties and possessions and dividing them among all according to whoever had need. Lk 2.44-45
\end{frame}

\begin{frame}
  This wasn't a forced practice, though.
  They did so willingly, as the rebuke to Ananias and Sapphira illustrates:\\
  Was it not remaining yours, and when sold, was it not in your authority? Acts 5.4\\
\end{frame}

\begin{frame}
  The early Church, while sharing all things was \emph{not} socialist.
  By the same token, it did not hold private property sacrosanct.
  There was enough pressure to share and help others, that Ananias and Sapphira felt the pressure to lie.
\end{frame}

\begin{frame}
  In Matthew 25, when Jesus recounts the final judgment, people are judged by how they treated the poor, needy, and helpless.
  James corresponds in James 2.14ff. by emphasizing that ``faith'' that doesn't tie to acts of mercy is ``dead.''
  It is, in fact, demonic.
  In I Cor. 13.2, the same point is made.
  Faith that doesn't share its possessions with others is not salvific, and the accumulation of wealth beyond what we need is sin.
\end{frame}

\begin{frame}
  These elements of the NT are not new.
  In the Old Testament, God's justice is not the punishment of wrong-doers but mercy to the poor.\pause
  \begin{itemize}
	\item Open your mouth and judge justice, defend the poor and the needy. Pv. 31.9\pause
	\item Do justice and love mercy and walk humbly with the \textsc{Lord} your God. Micah 6.8
  \end{itemize}
\end{frame}

\begin{frame}
  In the teaching God gave to Moses, there is still more.\pause
  \begin{itemize}
	\item Most famously, the Sabbath years and years of Jubilee when debs were released, and in the latter, slaves go free (Dt. 15.1-15, Lev 25)\pause
	\item Another instance, at harvest, the farmer was forbidden from making his full harvest; he had to leave gleanings for the poor(Lev 19.9-10)\pause
	\item Loaning money at interest is banned (Lev 25.35-37)
  \end{itemize}
\end{frame}

\begin{frame}
  These teachings of the Bible are everywhere.
  I've barely scratched the surface.
  I find it sad, that liberalism in its right-wing form has corrupted people so that they will argue passionately that Genesis must be taken literally but also argue that we can allegorize these passages away.
\end{frame}

\section{Digital vs. Material Economies}

\begin{frame}
  There is a fundamental difference between how resources work in the digital world vs. the material world.\\
  In the material world resources are:
  \begin{itemize}
	\item Finite.\pause
	\item Non-Replacable\pause
	\item Subject to scarcity
  \end{itemize}
\end{frame}

\begin{frame}
  Say a man had 3 oranges.
  Suppose I steal one.
  That man now has 2 oranges.
  Scarcity is impossible to avoid, and we cannot simply replicate the oranges.
\end{frame}

\begin{frame}
  Jesus' economics \emph{are} more moral, but they must be voluntary.
  And sometimes we really do have to choose betweeen feeding our children and feeding someone else.
  When these economics are translated into any form of government enforcement, they transform.
\end{frame}

\begin{frame}
  In order to try and put Jesus' practices into society-wide practice, we would need to change the following:\pause
  \begin{itemize}
	\item It must be violent.\pause
	\item It requires the decisions be made by the very people Jesus denounced.\pause
	\item It must be mandatory\pause
	\item It is generally pushed with envy and a rejection of authority.
  \end{itemize}
\end{frame}

\begin{frame}
  Jesus' teachings are likened to socialism.
  The end result is that it always kills people, and not in small quantities.
  Starvation, concentration camps, and the like.

  Jesus teachings cannot be translated directly as governmental policy.
\end{frame}

\begin{frame}
  The digital realm is \emph{quite} different. Scarcity doesn't naturally exist.\pause
  \begin{itemize}
	\item Objects have arbitrary and conceptual limits.\pause
	\item They are indefinitely replicable.\pause
	\item They are fundamentally replaceable.
  \end{itemize}
\end{frame}

\begin{frame}
  Ethical problems that challenge the application of Jesus' teachings in material do not exist for the digital.
  \begin{columns}
	\column{.4\textwidth}
	  My son is starving.
	  I have one small piece of fruit.
	  I can give it to him or another starving child.
	  I do not have any other food.
	\column{.4\textwidth}
	  My son lacks a program needed to complete his classes.
	  Another child lacks a program needed to complete his classes.
	  I can give my son a copy and the other child.
  \end{columns}
\end{frame}

\begin{frame}
  Jesus teachings may be easily applied literally.
  The ease in which they can be applied renders the application not even praiseworthy.
  There is no cost in doing so, save in choosing not to extract cost from others.
\end{frame}

\begin{frame}
  This applies to all forms of digital information.
  Programs, books, music, art, are all just information.
  Everything true about one is true about the other regarding our ability to copy them.
\end{frame}

\begin{frame}
  Objection: How woul people make a living?\pause
  \begin{itemize}
	\item People were able to make a living without restricting others' use of their intellectual property up until very recently in history.\\ \pause
	  They were still able to create and live off of it.\pause
	\item You always have the ability to utilize your creative ability to provide services:\pause
	  \begin{itemize}
		\item Books: You can provide lectures on your topic.\pause
		\item Programs: You can provide support or give an ongoing service with it.\pause
		\item Music: Playing live is always possible.\pause
		\item Other art: You can always compose or write on demand.\pause
	  \end{itemize}
	\item There will always be sponsors who wish to pay for particular things to be created.\pause
	\item Crowdfunding has now become a commonplace
  \end{itemize}
\end{frame}

\begin{frame}
   The negative features of much of the popular software is such that the makers forbid their children from using it.
   For example, Steve Jobs wouldn't let his children have a cell phone.
\end{frame}

\begin{frame}
  Proprietary software preys upon the poor.\pause
  \begin{itemize}
	\item It is often prohibitively expensive.\pause
	\item ``Free'' apps spy on their users and manipulate them into covetousness.\pause
	\item It is \emph{designed} to lose usefulness over time.\pause
	\item It is designed to put them into a walled garden and force them into it further.
  \end{itemize}
\end{frame}

\begin{frame}
  Negative features abound for the poor.
  It is not just lack of quality, but ``features'' that must be put in place deliberately and exploitative.
\end{frame}

\begin{frame}
  NOTE:\\
  It is \textbf{not} the technology of phones or computers that is so harmful.
  It is the software that is on them, and after that, how we use it.
  These uses and philosophies all have endgames that we need to consider.
  End games need not always be planned at the start.
\end{frame}

\begin{frame}
  The end game of proprietary software is the elimination of most physical property.\pause
  \begin{itemize}
	\item Most vehicles, appliances, et al. now have embedded software.\pause
	\item The licenses on these vehicles prohibit changing, altering, or replacing the software.\pause
	\item They now often include remote deactivation switches.\pause
	\item The push is to eliminate the ability of shade tree mechanics to work on vehicles.\pause
	\item John Deere, Ford, and others are now pushing an argument that since the hardware cannot run without the software, it cannot be replaced, and owners cannot own the software, they do not have a right to repair or possibly a right to own the vehicles.
  \end{itemize}
\end{frame}

\begin{frame}
  These arguments, if they stand, will apply to more than vehicles.
  All hardware with softare would be subject to it.
  This would end economic independence for people under several conditions.\\ \pause
  That \emph{is} the intention.
\end{frame}

\begin{frame}
  More positively, FOSS enables people to cooperate to help each other.
  Programs and information can be taylored for the interests of specific groups.
  Perseus, for instance, doesn't have a parallel in the proprietary world.
  Emacs has no parallel.
\end{frame}

\begin{frame}
  Costs can also be shared.
  This is published on LBRY.
  The server load is distributed among users so that it doesn't require an expensive server.
  Organizations like Youtube inherently lose money due to those serverloads.
  LBRY, however, not only doesn't have that serverload but can survive without the company.
\end{frame}

\begin{frame}
  One thing that has hurt Christian action and piety is the off-loading of responsibility to nations and large organizations.
  These are still possible with FOSS.
  We can also take anything and take personal control of it.
\end{frame}

\section{Conclusion}

\begin{frame}
  I don't mean to argue that we should use it exclusively; we can't as it stands.
  I mean to argue that it should be preferred when we can choose.
  And if we cannot choose something better, when it costs so little, how can we be trusted when it costs much.
\end{frame}

\begin{frame}
  We can also look to our children.
  What kind of world will we leave them?
  We have some control, and we'll be accountable for it.
\end{frame}

\begin{frame}
  FOSS also provides a model for other media.
  We can disconnect from the dragon's economics and embrace the economics of Jesus here.
  The principles can work for all forms of ``intellectual property.''
\end{frame}

\begin{frame}
  The closed method is the method of the dragon.
  It atomizes us and seeks for us to defend ``ours.''\\ \pause
  Christ's method is to provide a common tree of life for the healing of humanity.
  It seeks to bind us through cooperation and love.\\ \pause
  While I cannot say we must use a free software approach to our works, we cannot use Christian theology to justify the alternative.
\end{frame}

\begin{frame}
  Traditionalists say all the time, ``Reject the modern world.''
  This is a place we can.
  We can go back to premodern beliefs about authorship and creativity.
  If we can't even do that, when there is a thriving movement to support us, then we cannot do it anywhere else.
\end{frame}

\end{document}
