\documentclass{beamer}
\usepackage[T1]{fontenc}
\usepackage[utf8]{inputenc}
\usepackage{pslatex}
\usepackage[greek.polutoniko,hebrew,english]{babel}
\usepackage{cjhebrew}
\usetheme{Berkeley}
\usecolortheme{albatross}

\title{Christianity and Free Software Pt. 2}
\author{Kenneth Gardner}

\begin{document}

\maketitle

\begin{frame}
  There are two models for creativity today:\pause
  \begin{itemize}
	\item The standard model: The creator/company owns the idea, and it can only be used in ways they have licensed a person to use.\pause
	\item Intellectual works, when distributed, may be copied and modified by those who received them.
  \end{itemize}
\end{frame}

\begin{frame}
  They are ``copyright'' vs. ``copyleft'' for shorthand.
\end{frame}

\begin{frame}
  Another vantage for Christianity and free software is the composition of the Scripture and hymns.
  I take the following as axiomatic:\pause
  \begin{itemize}
	\item The process by which Scripture was composed should form a normative view to how we create things.\pause
	\item We should avoid adopting ethics that would render the very \emph{acts} of composing Scripture immoral.
  \end{itemize}
\end{frame}

\begin{frame}
  There are obvious limits to this.\pause
  \begin{itemize}
	\item We cannot use the technology as a model.\pause
	\item We cannot dismiss abuses in the old system.
  \end{itemize}
\end{frame}

\begin{frame}
  The argument goes that if it is immoral to copy or modify intellectual works, then the Bible is an inherently immoral document.
\end{frame}

\section{Synoptic Problem}

\begin{frame}
  The Synoptic problem is the problem of how the Gospels of Matthew, Mark, and Luke relate to one another.\pause
  \begin{itemize}
	\item Swaths of text appear to be copied verbatim between the three.\pause
	\item Did they modify each other, or were they all modifying a previous Gospel?\pause
  \end{itemize}
  It is evident there is a relation between the Gospels so that fully independent composition is not an option.
\end{frame}

\begin{frame}
  First we know that there were several editions floating around, and that at least Luke felt the need to write a better version:
  \begin{quote}
	Since many have set their hand to put in order an account concerning the things that have been fulfilled among us,
	just as those who were eyewitnesses and servants of the word handed it down to us,
	it seemed good to me also, having followed from it from the start accurately, to write for you an orderly account, O most excellent Theophilus
	so that you may know with certainty concerning the things you were taught. Lk 1.1-4
  \end{quote}
\end{frame}

\begin{frame}
  Greek is also a language where word order matters very little.
  So I could write \textgreek{ἄνθρωποι λέγουσι τοὺς μύθους τοῦ Δῖος}, ``People are telling the myths of Zeus.'' as\pause
  \begin{itemize}
	\item \textgreek{ἄνθρωποι λέγουσι τοὺς τοῦ Δῖος μύθους}\pause
	\item \textgreek{λέγουσι τοὺς τοῦ Δῖος μύθους ἄνθρωποι}\pause
	\item \textgreek{λέγουσιν ἄνθρωποι τοὺς τοῦ Δῖος μύθους}\pause
	\item \textgreek{ἄνθρωποι τοὺς τοῦ Δῖος μύθους λέγουσι}\pause
	\item And other ways
  \end{itemize}
\end{frame}

\begin{frame}
  They have extensive verbatim agreement ranging anywhere from 50\% of words to 90\% of their common material, but\pause
  \begin{itemize}
	\item Jesus taught in Aramaic\pause
	\item Greek has multiple words for things, just like English.\pause
	\item The word order is flexible as seen above
  \end{itemize}
\end{frame}

\begin{frame}
  The overall order of events is the same.
  Compare them with John's Gospel.
\end{frame}

\begin{frame}
  They maintain several of the same editorial comments.
  For example, ``let the reader understand'' is added in the same way and same place in the Olivet Discourse in all three (Mt. 24.15 in both Mk 13.14)
\end{frame}

\begin{frame}
  Mark and Luke will agree against Matthew.
  Mark and Matthew will agree against Luke.
  Luke and Matthew do not often agree against Mark.
\end{frame}

\begin{frame}
  The evidence gives us this model:\pause
  \begin{itemize}
	\item Mark was composed first.\pause
	\item Independent documents usually called Q were composed.\pause
	\item Documents used by Matthew, M, were composed.\pause
	\item Documents used by Luke, L, were composed.\pause
	\item Greek Matthew composed his Gospel by using Mark, Q, and M.\pause
	\item Luke composed his Gospel using Mark, Q, and L.
  \end{itemize}
\end{frame}

\begin{frame}
  The Gospels are, thus, heavily interrelated.
  They were copying and modifying preceding works in ways that we would consider immoral today.
\end{frame}

\section{Ezras}

\begin{frame}
  There are several books of Ezra.
  We have apocalypses and narratives.
\end{frame}

\begin{frame}
  They also come to us under different names.
  In the Septuagint 1 Esdras is unique, 2 Esdras is Ezra and Nehemiah.
  There is a 3 Esdras added to Slavonic Bibles that is an Apocalypse.
  In the Vulgate 1 Esdras is Ezra, 2 Esdras is Nehemiah, 3 Esdras is the LXX 1 Esdras.
  The Apocalypse is 4 Esdras.
  I follow the LXX here.
\end{frame}

\begin{frame}
  1-2 Chronicles reworks the history in 1-2 Sam. and 1-2 Kings, but it is its own thing.
\end{frame}

\begin{frame}
  1 Esdras begins with the chapters 35 and 36 of 2 Chronicles.
  Several questions are immediately apparent:\pause
  \begin{itemize}
	\item Was 1 Esdras intended to continue Chronicles?\pause
	\item Was it by the same author?\pause
	\item Was it a revision by a later editor to complete a story he felt unfinished?
  \end{itemize}
\end{frame}

\begin{frame}
  Then 2 Esdras reworks 1 Esdras.
  The bulk of the Ezra portion includes material common 1 Esdras to Ezra-Nehemiah.
\end{frame}

\begin{frame}
  Ezra-Nehemiah includes two portions.
  \begin{itemize}
	\item Ezra
	\item Nehemiah
  \end{itemize}
\end{frame}

\begin{frame}
  These were probably originally separate books.\pause
  \begin{itemize}
	\item They have their own plot lines.\pause
	\item They have few characters in common.
  \end{itemize}
\end{frame}

\begin{frame}
  However, they form a coherent pairing:\pause
  \begin{itemize}
	\item They share the same themes.\pause
	\item They share similar timelines.\pause
	\item They have a similar broad shape.\pause
	\item They are transmitted together.
  \end{itemize}
\end{frame}

\begin{frame}
  Ezra-Nehemiah, as it stands, forms a comprehensible book that was probably one book.
  It also reworked previous books and editions.
\end{frame}

\begin{frame}
  The question we have to ask about 1 Esdras and Ezra-Nehemiah is ``Which one copied which?''\pause
  \begin{itemize}
	\item 1 Esdras is a revision of Ezra-Nehemiah.\pause
	\item Ezra-Nehemiah is a revision of 1 Esdras.\pause
	\item Both may be dependent on the same earlier material.
  \end{itemize}
\end{frame}

\begin{frame}
  There are inevitable stages of redaction:\pause
  \begin{itemize}
	\item 2 Esdras used earlier material to compose its text\pause
	\item Proto-Ezara and Proto-Ezra used to make Ezra-Nehemiah.\pause
	\item Modern Bibles edit Ezra-Nehemiah to create two books, Ezra and Nehemiah.\pause
  \end{itemize}
  These facts remain even if we treat Ezra and Nehemiah as almost entirely replicating PE and PN.
\end{frame}

\section{Daniel}

\begin{frame}
  Daniel is one of the more curious examples of promiscuously combining sources.
\end{frame}

\begin{frame}
  The Masoretic version three linguistic divisions:\pause
  \begin{itemize}
	\item The first part of the book is Hebrew.\pause
	\item It switches to Aramaic in 2.4.\pause
	\item It switches back to Hebrew after 7.28
  \end{itemize}
\end{frame}

\begin{frame}
  Daniel mixes first and third person narratives.
\end{frame}

\begin{frame}
  It has portions composed of stories, and portions composed of surreal dreams.
\end{frame}

\begin{frame}
  Even without getting to additions in the Septuagint, Daniel is a composite book redacting at least two earlier sources.
\end{frame}

\begin{frame}
  Most Christians in the world use a Daniel that has additions from the Septuagint.
  This adds Greek to the mix of languages the book is written in.\pause
  \begin{itemize}
	\item There are two Greek stories appended or prefixed, Bel and the Dragon as well as Susanna.\pause
	\item An expansion to the story of the fiery furnace is inserted into Dan 3.
  \end{itemize}
\end{frame}

\begin{frame}
  The Greek comes in two editions:\pause
  \begin{itemize}
	\item The Old Greek\pause
	\item Theodotion
  \end{itemize}
\end{frame}

\begin{frame}
  The DSS make the picture even more complicated.
  There are no records of the Septuagint additions.
  There are however several fragments, and they show a mix of readings between the OG and MT.
\end{frame}

\section{Jeremiah}

\begin{frame}
  There are, essentially, two books of Jeremiah.\pause
  \begin{itemize}
	\item The Masoretic Text\pause
	\item The Septuagint
  \end{itemize}
\end{frame}

\begin{frame}
  The LXX is shorter by about 20\%. The text is in a different order.
\end{frame}

\begin{frame}
  It is not clear which one was earlier.
\end{frame}

\begin{frame}
  Three individual mss of Jeremiah are in Qumran.\pause
  \begin{itemize}
	\item The largest manuscript agrees with the MT.\pause
	\item The other two fragmentary mss show LXX readings.
  \end{itemize}
\end{frame}

\begin{frame}
  Both editions existed side by side with apparently no difficulty.
\end{frame}

\section{Implications}

\begin{frame}
  There are many, many other examples of this sort of authorship in the Bible from Proverbs, to the Woman Caught in Adultery, to Ezekiel.
\end{frame}

\begin{frame}
  These facts cause difficulty for modern faith, because our understanding of creativity is not compatible with historic human creativity.
\end{frame}

\begin{frame}
  The most basic definition of FOSS posits the four freedoms:\pause
  \begin{enumerate}
	\item The right to see the source\pause
	\item The right to modify the source.\pause
	\item The right to redistribute it.\pause
	\item The right to distribute your modifications.\pause
  \end{enumerate}
  And denying people these rights is immoral.
\end{frame}

\begin{frame}
  This model of software development can also be applied to any form of human intellectual endeavor.
\end{frame}

\begin{frame}
  It also better represents what we see in how the biblical text was authored.
\end{frame}

\begin{frame}
  Scripture is not just a series of propositions, but a whole way of seeing the world, values, habits, idioms, and many other non-propositional features.
  The nature of creativity, inspiration, and the like belong in that rubric.
\end{frame}

\begin{frame}
  The copyleft approach to intellectual ``property'' better resembles what we see in the actual biblical text.
  Accordingly, copyleft approaches should be given priority over copyright approaches.
\end{frame}

\section{Objections}

\begin{frame}
  There are basic objections people can raise, and they need to be addressed.
  Many of these I anticipate.
  There aren't many people promiscuously mixing free software and Christian beliefs.
\end{frame}

\begin{frame}
  The first objection:\\

  This format allows for anybody to edit the Bible as they see fit.
\end{frame}

\begin{frame}
  It already happens.\pause
  \begin{itemize}
	\item There are no laws against editing the Bible.\pause
	\item Varying levels of minor changes are acceptable in to Christians.
  \end{itemize}
\end{frame}

\begin{frame}
  I have already mentioned Ezra-Nehemiah.
  The fact that splitting a book into two isn't considered editing is evidence itself that we have a threshold in which it is allowed.
\end{frame}

\begin{frame}
  Another common example is the translation of ``hell.''
  Neither the word, nor arguably, its equivalents appear often in the Bible.\pause
  \begin{itemize}
	\item ``Hades'' and ``Sheol'' are realms of the dead, both good and evil.\pause
	\item ``Gehenna'' is ``The Valley of Hinnom'' and is a literal place.\pause
	\item ``Tartarus'' is the fold of Hades where rebel divinities are tortured.
	  This is our only close parallel.
  \end{itemize}
\end{frame}

\begin{frame}
  The NIV translates \textgreek{par'adosis} as ``teaching'' or the like whenever it is doctrinally inconvenient.
  It is one of the most common NT translations.
\end{frame}

\begin{frame}
  Most modern Bibles are eclectic texts trying to reconstruct an earlier form and, as a whole, do not represent the reading in any manuscript.
\end{frame}

\begin{frame}
  Even our formats are novel.
  Chapter and verse divisions, as well as our subheadings and paragraph divisions do not exist in the text.
  They can change the meaning.
\end{frame}

\begin{frame}
  These seem minor, because we accept them.
  They are objectively changes to the text.
\end{frame}

\begin{frame}
  A biblical book and its text form are ``canonical'' if they are accepted and read in churches.
\end{frame}

\begin{frame}
  In antiquity, the middle ages, and the modern world we have people editing the Bible in ways that are deemed unacceptable.
  We reject them in exactly the same ways.
  Our modern approaches to authorship and IP inventions have not made any difference on this issue.
\end{frame}

\begin{frame}
  Another objection is that we have changed several things from the Bible to the modern era.
  Slavery is the easiest example.
\end{frame}

\begin{frame}
  The biblical text presupposed slavery in most places in much the same way as it presupposes what I'm arguing.
  It even has laws regulating it.
  We do not accept slavery as a moral option in the modern era and feel no moral difficulty in this.
\end{frame}

\begin{frame}
  Slavery changed due to theological pressure.\pause
  \begin{itemize}
	\item Philemon includes an anti-slavery theme.\pause
	\item Revelation also includes some where Babylon's trade goods include the ``bodies and lives of people.''\pause
	\item The rhetoric of salvation is tied up with liberation from slavery from Exodus through the end of the NT.\pause
	\item The Image of God confers a dignity that is not compatible with slavery.
  \end{itemize}
\end{frame}

\begin{frame}
  No such thematic pressures exist to support intellectual property or its ethics.
  They run up against biblical ethics as I demonstrated in my first video.
\end{frame}

\section{Conclusion}

\begin{frame}
  The argument about the Bible having ``versions'' and not being reliable does not reflect reality.
  However, the truth of the text would be just as offensive as what the objectors imagine.
  Most modern conceptions of authorship, its perks, and how we respond to intellectual works are simply not compatible with the Bible.
\end{frame}

\begin{frame}
  The FOSS movement, copyleft, is in substantial conformity with the production process of the Bible.
  The truth about the Bible would not raise any discomfort if one holds it.
  It doesn't even require apologetics about the authorship.
\end{frame}

\begin{frame}
  Christians should leave copyright behind and embrace copyleft.
  We should publish as much of our material in a manner like FOSS as possible.
\end{frame}

\end{document}
